\documentclass[11pt, a4paper, twocolumn]{article}
\usepackage[left=14mm,text={183mm, 252mm},top=23mm]{geometry}
\usepackage[T1]{fontenc}
\usepackage[utf8]{inputenc}
\usepackage[czech]{babel}
\usepackage{amsthm}
\usepackage{amsmath}
\usepackage{lmodern}

\hyphenation{Mea-ly-ho}


\begin{document}

\begin{titlepage}
\begin{center}
\thispagestyle{empty}
%\Huge
\textsc{\Huge Vysoké učení technické v Brně}\huge\\[0.5em]
\textsc{\huge Fakulta informačních technologií}\\
\vspace{\stretch{0.382}}
{\LARGE Typografie a publikování – 2. projekt}\large\\[0.6em]
{\LARGE Sazba dokumentů a matematických výrazů}\\
\vspace{\stretch{0.618}}

\end{center}
{\Large 2024 \hfill Hugo Bohácsek (xbohach00)}
\end{titlepage}
\newpage
%\label{strana1}
\setcounter{page}{1}
%\twocolumn
\section*{Úvod}
V této úloze si vysázíme titulní stranu a kousek matematického textu, v němž se vyskytují například Definice \ref{def:definice1} nebo rovnice (\ref{rovnica2}) na straně \pageref{def:definice1}. Pro vytvoření
těchto odkazů používáme kombinace příkazů \verb|\label|,
\verb|\ref|, \verb|\eqref| a \verb|\pageref|. Před odkazy patří nezlomitelná mezera. Pro zvýrazňování textu se používají
příkazy \verb|\verb| a \verb|\emph|.

Titulní strana je vysázena prostředím \texttt{titlepage}
a~nadpis je v optickém středu s využitím \textit{přesného} zlatého řezu, který byl probrán na přednášce. Dále jsou
na titulní straně čtyři různé velikosti písma a mezi
dvojicemi řádků textu je použito řádkování se zadanou relativní velikostí 0,5\,em a 0,6\,em\footnote{Použijte správný typ mezery mezi číslem a jednotkou.}.
\section{Matematický text}
Matematické symboly a výrazy v plynulém textu jsou
v prostředí \texttt{math}. Definice a věty sázíme v prostředí
definovaném příkazem \verb|\newtheorem| z balíku \texttt{amsthm}.
Tato prostředí obracejí význam \verb|\emph|: uvnitř textu
sázeného kurzívou se zvýrazňuje písmem v základním řezu. Někdy je vhodné použít konstrukci \verb|${}$|
nebo \verb|\mbox{}|, která zabrání zalomení (matematického) textu. Pozor také na tvar i sklon řeckých písmen:
srovnejte \verb|\epsilon| a \verb|\varepsilon|, \verb|\Xi| a \verb|\varXi|.
\newtheorem{defn}{Definice}
\begin{defn}\label{def:definice1}
v\emph{Konečný přepisovací stroj} neboli \emph{Mealyho automat} je definován jako uspořádaná pětice tvaru $M = \mathrm{(}Q,\varSigma,\varGamma,\delta,q_0\mathrm{)}$, kde:
\begin{itemize}
    \item[$\bullet$] Q je konečná množina \emph{stavů},
    \item[$\bullet$] $\varSigma$ je konečná \emph{vstupní abeceda},
    \item[$\bullet$] $\varGamma$ je konečná \emph{výstupní abeceda},
    \item[$\bullet$] $\delta$ \emph{:} $Q \times \varSigma \rightarrow Q \times \varGamma$ je totální \emph{přechodová funkce},
    \item[$\bullet$] $q_0 \in Q$ je \emph{počáteční stav}.
    %δ : Q × Σ → Q × Γ je totální přechodová funkce,
\end{itemize}
%tvaru M = (Q, Σ, Γ, δ, q0), kde:
\end{defn}
\subsection{Podsekce s definicí}
Pomocí přechodové funkce $\delta$ zavedeme novou funkci $\delta^\ast$
pro překlad vstupních slov $u \in \Sigma^\ast$ do výstupních slov $w \in \Gamma^\ast$.
%\newtheorem{defn2}{Definice}
\begin{defn}
Nechť $M = \mathrm{(}Q, \varSigma, \varGamma, \delta, q_0\mathrm{)}$ je Mealyho automat. \emph{Překládací funkce} $\delta^\ast : Q \times \varSigma^\ast \times \varGamma^\ast \rightarrow \varGamma^\ast$ je pro každý stav $q \in Q$, symbol $x \in \varSigma$, slova $u \in \varSigma^\ast$, $w \in \varGamma^\ast$ definována rekurentním předpisem:
\begin{itemize}
    \item[$\bullet$] $\delta^\ast(q, \varepsilon, w) = w$
    \item[$\bullet$] $\delta^\ast(q, xu, w) = \delta^\ast(q^\prime, u, wy)$, kde $(q^\prime, y) = \delta(q,x)$
\end{itemize}
\end{defn}

\vfill\eject % remove meeeeeeeeeeeeeeeeeeeeeeeeeeeeeeeeeeeeeeeeeeeeeeeeeeeeeee

\subsection{Rovnice}
Složitější matematické formule sázíme mimo plynulý
text pomocí prostředí \texttt{displaymath}. Lze umístit i více
výrazů na jeden řádek, ale pak je třeba tyto vhodně
oddělit, například pomocí \verb|\quad|, při dostatku místa i~\verb|\qquad|.
$$g^{a_n} \notin A^{B^n}\qquad y^1_0 - \sqrt[5]{x + \sqrt[7]{y}}\qquad x \ge y^2 \geq y^3$$

Velikost závorek a svislých čar je potřeba přizpůsobit jejich obsahu. Velikost lze stanovit explicitně, anebo pomocí \verb|\left| a \verb|\right|. Kombinační čísla sázejte makrem \verb|\binom|.
$$\left| \bigcup P \right| = \sum\limits _{\emptyset \neq X\subseteq P} (-1)^{\left|X\right|-1}\left|\bigcap X\right|$$\bigskip
$$F_{n+1}=\binom{n}{0} + \binom{n-1}{1} + \binom{n-2}{2} + \cdots + \binom{\left\lceil \frac{n}{2} \right\rceil}{\left\lfloor \frac{n}{2} \right\rfloor}$$

V rovnici (1) jsou tři typy závorek s různou \emph{explicitně} definovanou velikostí. Obě rovnice mají svisle zarovnaná rovnítka. Použijte k tomu vhodné prostředí.
\setcounter{equation}{0}
\begin{eqnarray}
\biggl(\Bigl\{b\otimes [c_1 \otimes c_2] \circ a\Bigl\}^{\frac{2}{3}}\biggr) & = & log_zx \\
\label{rovnica2}\int_{a}^{b} f(x)\,\mathrm{d}x & = & -\int_{b}^{a}f(x)\,\mathrm{d}y
\end{eqnarray}
V této větě vidíme, jak se vysází proměnná určující limitu v běžném textu: $\lim_{x\rightarrow\infty}f(m)$. Podobně je to i s dalšími symboly jako $\bigcup_{N\in \mathcal{M}}N$ či $\sum_{i=1}^{m}x_{i}^{2}$. S vynucením méně úsporné sazby příkazem \verb|\limits| budou vzorce vysázeny v podobě $\displaystyle{\lim_{m\rightarrow\infty}f(m)}$ a $\sum\limits_{i=1}^{m}x_{i}^{2}$.

\section{Matice}
Pro sázení matic se používa prostředí \texttt{array} a závorky s výškou nastavitelnou pomocí \verb|\left|, \verb|\right|.
\[
D=
 \left| 
\begin{array}{cccc}
     a_{11} & a_{12} & \cdots & a_{1n}\\
     a_{21} & a_{22} & \cdots & a_{2n}\\
     \vdots & \vdots & \ddots & \vdots\\
     a_{m1} & a_{m2} & \cdots & a_{mn} 
\end{array}
\right|
=
\left|
\begin{array}{cc}
    x & y\\
    t & w
\end{array}
\right|
= xw - yt
\]

Prostředí \texttt{array} lze úspěšně využít i jinde, například na pravé straně následujíci rovnosti.
\[
\binom{n}{k} = 
\Biggl\{
\begin{array}{ll}
    \frac{n!}{k!(n-k)!} & \text{pro }0 \leq k \leq n\\
    0 & \text{jinak}
\end{array}
\]
\end{document}
